\documentclass[10pt]{ctexart}
\usepackage{amsmath,amsfonts,amssymb,amsthm}
\usepackage[a4paper,left=2.5cm,right=2.5cm,top=2cm,bottom=2cm]{geometry}
\usepackage{graphicx}
\usepackage{float,bm}
\usepackage{subfigure}
\author{2000012425 张弛}
\title{阿贝成像原理和空间滤波}
\CTEXsetup[format={\Large\bfseries}]{section}
\newtheorem{fuck}{证明}
\newtheorem{proposition}{命题}
\begin{document}
\maketitle
\noindent\textbf{摘\quad 要:}
本实验验证阿贝成像原理并进行空间滤波实验.观察不同物的频谱形态.通过遮挡部分频谱来进行滤波,观察像的变化.进行$\theta$调制实验.观察阿贝成像原理的卷积现象.\\
\noindent\textbf{关键词:}
傅里叶光学\quad 阿贝成像\quad 衍射
\section{实验原理暨预期}
\subsection{衍射积分与傅里叶变换}
光路如图.实验使用$4f$光路,以下仅对$4f$情形分析.
\begin{figure}[H]
\centering
\includegraphics[height=3.5cm]{4f.png}
\caption{实验光路}
\end{figure}
先看$\Sigma_g$到$\Sigma_p$,考虑平行光入射,使用傍轴情形的基尔霍夫衍射积分
$$U_p(x_p,y_p)=-\frac{i}{\lambda}\iint \widetilde{U_0}t(x_g,y_g)\frac{e^{ik(L_0(x_p,y_p)-x_g\sin{\theta_1}-y_g\sin{\theta_2})}}{f}dx_gdy_g.$$
其中$L_0(x_p,y_p)$是物在光轴上的点到频谱面上点$(x_p,y_p)$的光程,$\theta_1$与$\theta_2$为对应于频谱面上点$(x_p,y_p)$的衍射角.由于物在焦面上,因此$L_0(x_p,y_p)$为常数,任取不影响成像,不妨
$$L_0(x_p,y_p)=0.$$
由几何光学可得衍射角$\theta_1,\theta_2$与$x_p,y_p$的关系
$$\sin{\theta_1}=\frac{x_p}{f},\sin{\theta_2}=\frac{y_p}{f}.$$
则得到频谱波前
$$U_p(x_p,y_p)\propto \iint t(x_g,y_g)e^{-ik(x_p/f*x_g+y_p/f*y_g)}dx_gdy_g.$$
是傅里叶变换的形式.记为
$$U_p(x_p,y_p)\propto \mathcal{F}[t(x_g,y_g)]=T(x_p,y_p).$$
由于只关注图像的形状特征,因此略去波长相关倍率,仅取形式.

由光路的可逆性,$\Sigma_p$到$\Sigma_h$的过程是$\Sigma_g$到$\Sigma_p$的逆过程,则有
$$U_h(x_h,y_h)\propto \mathcal{F}^{-1}[T(x_p,y_p)]=t(x_g,y_g).$$
同样忽略掉了倍率关系.

此后的讨论中直接用原函数$t$与象函数$T$表示相关波前,在傅里叶变换时忽略掉倍率关系.

衍射积分与傅里叶变换数学本质相同,而傅里叶变换的意义更加直观,故此后的讨论中摒弃衍射积分的观点.
\subsection{观察频谱分布}
以下讨论中,衍射纹、象函数、频谱指的是同一回事,物与原函数指的也是同一回事.作出实验预期的同时,也对实验现象作出了解释.每小节符号重置.
\subsubsection{正$n$边形}
先看一下狭缝,写出原函数
$$u(x,y)=\eta(a-|x|).$$
则象函数
$$U(x',y')=\delta(y')\frac{\sin{(ax')}}{x'}.$$
忽略掉了一些与波长相关的缩放倍率,以下均忽略.

考虑形式如下的原函数
$$g(x,y)=\eta(a-|x|)f(x,y).$$
记
$$\mathcal{F}\{f\}=F(x',y').$$
则由卷积定理
$$\mathcal{F}\{g\}=U*F=\int\frac{\sin{[a(x'-x)]}}{x'-x}F(x,y')dx$$

对于任意一个正$n$边形,不论奇偶,总能看作$s$个不同方向的狭缝的公共区域,对于偶数为$n/2$个,对于奇数为$n$个.这些狭缝的原函数分别记为
$$u_1,u_2...u_s.$$
则正$n$边形的原函数可以写成
$$t=\displaystyle\prod_{i=1}^{s}u_i.$$
则由卷积定理,正$n$边形的象函数
$$T=U_1*U_2*...*U_s.$$

先考虑$s=2$情形
$$U_1*U_2=\int\frac{\sin{[a(x'-x)]}}{x'-x}U_1dx.$$
由夫琅禾费衍射事实可以知道,$U_1$在原点附近的起伏远比其他部分大,因此不妨认为
$$U_1=\delta(\bm{x}).$$
则
$$U_1*U_2\approx\frac{\sin{(ax')}}{x'}.$$
呈现出来的图像是,在$U_2$的$x$轴方向上出现了与$U_1$相同的频谱分布,而$U_1$的$x$轴方向的频谱分布应当与$U_2$的频谱分布量级相通,于是就是两组径向的明暗交替的类似于狭缝衍射的纹样.

随着$s$增大,可以定性地分析认为,每卷积一个狭缝,就在这个狭缝的垂直方向上多一组径向的明暗交替的狭缝衍射纹,但明暗周期以及强度分布逐渐发生畸变.

所以结论是,正$n$边形的频谱呈现为垂直于$n$条边的多组径向的明暗交替的衍射纹.
\subsubsection{单方孔与矩孔}
卷积的结果比较简单,象函数
$$U_1*U_2=\frac{\sin{(ax')}}{x'}\frac{\sin{(ay')}}{y'}.$$
很标准的夫琅禾费矩孔衍射.
\subsubsection{单圆孔}
可以视为$s\rightarrow\infty$情形的正$n$边形,则可以定性地判断,在各个方向上都有明暗交替的衍射纹样,且旋转对称.

这与夫琅禾费圆孔衍射的预期一致.
\subsubsection{正三角孔}
是$s=3$的情形,则可以定性地判断,在垂直于三条边的方向上有三组相同的明暗交替的衍射纹样,成6枝放射状,并且有轴对称.
\subsubsection{五角星孔}
虽然不是正$5$边形,但可以切割成中心的小正$5$边形与边上$5$个小$3$边形,由傅里叶变换的线性性质,可以分别计算贡献.每一个小$3$边形的每条边都与正$5$边形的某条边平行,因此在衍射纹的方向分布上与正$5$边形一致.明暗间距不论.

则可以定性地判断,在五角星$5$个角的方向都有明暗交替的衍射纹样,成10枝放射状,并且有轴对称.
\subsubsection{方孔方阵与方孔密排}
设中心在原点的小方孔的原函数为$f(\bm{x})$,则方孔方阵的原函数可以写成
$$t(\bm{x})=\displaystyle\sum_{i=1}^n f(\bm{x}-\bm{x_i})=\int f(\bm{x'})g(\bm{x}-\bm{x'})d^3\bm{x'}.$$
其中
$$g(\bm{x})=\displaystyle\sum_{i=1}^n \delta(\bm{x}-\bm{x_i}).$$
是二维方阵点阵,不妨取从$(a,a)$到$(na,na)$的$n^2$个点,其象函数
$$\mathcal{F}\{g\}=\frac{1}{2\pi}\displaystyle\sum_{i,j=1}^n\delta(x-ia)\delta(y-ja)e^{-ix'x-iy'y}dxdy=c\frac{\sin{(nax'/2)}}{\sin{(ax'/2)}}\frac{\sin{(nay'/2)}}{\sin{(ay'/2)}}.$$
\begin{figure}[H]
\centering
\includegraphics[height=6cm]{2DS.png}
\caption{二维方阵点阵象函数}
\end{figure}
则由卷积定理,方孔方阵的象函数
$$T(\bm{x'})=\mathcal{F}\{f\}\mathcal{F}\{g\}.$$
也即方孔衍射与二维方阵点阵衍射的乘积,因此呈现出来的应当是如图基础上的强度受方孔衍射纹调制的图像.

对于方孔密排,$g(\bm{x})$应当改写,不妨将上述方阵的行数与列数改为$2n$,偶数行向$x$方向平移$a/2$,则可算得象函数
$$\mathcal{F}\{g\}=c\frac{\sin{(nx'a)}}{\sin{(x'a)}}\frac{\sin{(ny'a)}}{\sin{(y'a/2)}}(1+e^{-ix'a-iy'a/2}).$$
\begin{figure}[H]
\centering
\includegraphics[height=7cm]{2DM.png}
\caption{二维密排点阵象函数}
\end{figure}
则由卷积定理,方孔密排的象函数
$$T(\bm{x'})=\mathcal{F}\{f\}\mathcal{F}\{g\}.$$
也即方孔衍射与二维密排点阵衍射的乘积,因此呈现出来的应当是如图基础上的强度受方孔衍射纹调制的图像.
\subsubsection{圆孔方阵与圆孔密排}
同上分析,结果分别为强度受圆孔衍射调制的二维方阵点阵衍射纹与二维密排点阵衍射纹.

\subsection{空间滤波}
\subsubsection{一维光栅}
设狭缝原函数为$f(\bm{x})$,由前面的讨论知道
$$\mathcal{F}\{f\}=\delta(y')\frac{\sin{(ax')}}{x'}.$$
忽略掉了与波长相关的缩放倍率,也忽略掉了前面的因子,以下也忽略.原函数可以写成卷积的形式
$$t(x,y)=\displaystyle\sum_{i=1}^{N}f(\bm{x}-\bm{x_i})=\int f(\bm{x'})g(\bm{x}-\bm{x'})d^3\bm{x'}.$$
其中
$$g(\bm{x})=\displaystyle\sum_{i=1}^{N}\delta(\bm{x}-\bm{x'}).$$
则由卷积定理
$$T(x',y')=\mathcal{F}\{f\}\mathcal{F}\{g\}=\delta(y')\frac{\sin{[Nbx'/2]}}{\sin{(bx'/2)}}\frac{\sin{(ax')}}{x'}.$$
是普通的一维光栅的衍射纹.随后的成像过程是逆变换,略去系数
$$t(x,y)=\iint \delta(y')\frac{\sin{[Nbx'/2]}}{\sin{(bx'/2)}}\frac{\sin{(ax')}}{x'}e^{ix'x+iy'y}dx'dy'=\int \frac{\sin{[Nbx'/2]}}{\sin{(bx'/2)}}\frac{\sin{(ax')}}{x'}e^{ix'x}dx'.$$
由一维光栅的衍射性质可以知道,衍射波前的峰是尖锐的,能量主要集中在各个主峰处,因此逆变换的积分可以离散化为级数
$$t(x,y)=\displaystyle\sum_{i}\frac{\sin{[Nbk_i/2]}}{\sin{(bk_i/2)}}\frac{\sin{(ak_i)}}{k_i}e^{ik_ix}.$$
其中
$$k_i=\frac{2i\pi}{b}.$$

空间滤波,若仅通过0级衍射纹,则
$$t'(x,y)\approx aN.$$
常数,则成像为均匀亮斑.

若通过0级与$\pm 1$级衍射纹,则
$$t'(x,y)\approx aN+2\frac{\sin{(2\pi a/b)}}{2\pi/b}N\cos{(\frac{2\pi}{b}x)}.$$
应当已经形成了空间周期为$b$的纹样.

若遮挡$\pm 1$级衍射纹,则可以知道遮挡掉的是空间周期为$b$的成分,剩余的级数中最大的空间周期是$b/2$,也即
$$t'(x,y)=aN+2\frac{\sin{(4\pi a/b)}}{4\pi/b}N\cos{(\frac{4\pi}{b}x)}+....$$
成像为空间周期为$b/2$的纹样.

若遮挡$0$级衍射纹,则过滤掉的是级数中的常数项.由于原函数$t(x,y)$是恒正的,而级数中的常数项可以视为原函数的平均值,则在去掉平均值之后原函数$t'(x,y)$整体向负方向平移.若原函数$t(x,y)$中通光的部分更多,则平均值应当大于$1/2$,则在去掉平均值平移之后的$t'(x,y)$通光部分变为模较小的部分,而不通光的部分变为模较大的部分,呈现出衬比度反转的现象.若原函数$t(x,y)$中不通光的部分更多,则$t'(x,y)$通光部分依旧模更大,衬比度变小但不反转.

除了可能的衬比度反转,其余周期等特征不变.
\subsubsection{二维光栅}
频谱特征是显然的,不必写出具体表达式就可以定性地分析.

若仅通过最中心的$0$级衍射纹,则类比一维光栅的情形,呈现出来的$t'(x',y')$应该是常数波长,成像为均匀亮斑.

若用狭缝进行滤波,先考虑狭缝函数的逆变换
$$F(x',y')=\eta(a-|x'|),$$
$$f(x,y)=2\delta(y)\frac{\sin{(ax)}}{x}.$$
则对二维光栅衍射纹$T(x',y')$的狭缝滤波由卷积定理得
$$t'(x,y)=\mathcal{F}\{FT\}=f*t=\int 2\frac{\sin{a(x-x')}}{x-x'}t(x',y)dx'.$$
相当于将原来的像$t(x,y)$向狭缝的垂直方向加权滑动平均.

当仅通过中心水平部分的频谱时,相当于将像在竖直方向上滑动加权平均,平均的结果在水平方向的差异应该是明显的,呈现出来的应当是竖条纹.

当仅通过中心竖直部分的频谱时,相当于将像在水平方向上滑动加权平均,平均的结果在竖直方向的差异应该是明显的,呈现出来的应当是水平条纹.

当仅通过中心倾斜$45^\circ$部分的频谱时,相当于将像在另一倾斜方向上滑动加权平均,各处的平均结果应该是相同的,呈现出来的应当是均匀光斑.
\subsubsection{“光”字形二维光栅}
原函数可以视为光字形$f(x,y)$与二维光栅函数$g(x,y)$的乘积
$$t(x,y)=fg.$$
卷积定理
$$T(x',y')=G*F.$$
频谱应当为两者的彼此调制,光字形与二维光栅的总大小是同一大尺度的,而二维光栅的小网格是小尺度的,则频谱的大尺度特征应当与方孔方阵相同,而小尺度特征受光字形影响.

当仅通过中心的光斑时.由于光字形的信息体现在频谱的小尺度特征里,因此像应当保持光字形.而二维光栅的特征应当体现在频谱的大尺度特征中,当仅通过中心光斑时,只保留了二维光栅的低频特征,因此像会模糊.故呈现出来的是一个模糊的无网格特征的光字.

当缩小中心通过光斑时,主要的改变体现在频谱中光字形的信息的减少,故呈现出来的结果是光字形变模糊.

当仅通过中心下侧相邻光斑时.光字形特征得以保留,而二维光栅保留了反映空间周期的信息,故呈现出来的是一个有平行线特征的光字.

当缩小中心下侧相邻光斑时,光字形应当变模糊.
\subsubsection{十字光孔}
可以明显的感受到,原函数与方孔有着相似之处,都可以看作两条狭缝的正交,因此频谱特征应当与方孔衍射纹类似,但在频谱的小尺度特征中应当有十字的信息.

当遮挡掉频谱的较大范围的低频成分时,由于十字的信息体现在频谱的小尺度特征里,而在遮挡部分外有大量的小尺度特征,因此成像中的十字不变.在频谱的大尺度范围内,除去频谱中心大范围的低频成分之后,剩余的高频成分起伏很小,不妨视为常数,而
$$\mathcal{F}^{-1}\{1\}=2\pi\delta(x).$$
则可以估计成像呈现出来的是一个十字轮廓.
\subsection{$\theta$调制}
预期是显然的,不作分析.
\subsection{卷积现象}
原函数是两个不同光栅常数的二维光栅成角度的乘积.则由卷积定理可以断言,频谱的大尺度特征与光栅常数较小的二维光栅的频谱相同,小尺度特征与光栅常数较大的二维光栅的频谱相同.

没有具体计算的必要.
\section{实验}
\subsection{实验装置}
略.

光路图前面已经呈现.
\subsection{实验过程}
略.
\section{实验结果与数据处理}
\subsection{观察频谱分布}
\subsubsection{单方孔与矩孔}
成像,频谱分别如图.
\begin{figure}[H]
\begin{minipage}[t]{0.5\linewidth}
\centering
\includegraphics[width=0.8\textwidth]{3111.jpg}
\caption{单方孔像} 
\end{minipage}
\hfill
\begin{minipage}[t]{0.5\linewidth}
\centering
\includegraphics[width=0.8\textwidth]{3112.jpg}
\caption{单方孔频谱}
\end{minipage}
\end{figure}
\begin{figure}[H]
\begin{minipage}[t]{0.5\linewidth}
\centering
\includegraphics[width=0.8\textwidth]{3113.jpg}
\caption{矩孔像} 
\end{minipage}
\hfill
\begin{minipage}[t]{0.5\linewidth}
\centering
\includegraphics[width=0.8\textwidth]{3114.jpg}
\caption{矩孔频谱}
\end{minipage}
\end{figure}
标准的夫琅禾费矩孔衍射纹,符合前面的推理预期.
\subsubsection{单圆孔}
成像,频谱如图.
\begin{figure}[H]
\begin{minipage}[t]{0.5\linewidth}
\centering
\includegraphics[width=0.8\textwidth]{3121.jpg}
\caption{圆孔像} 
\end{minipage}
\hfill
\begin{minipage}[t]{0.5\linewidth}
\centering
\includegraphics[width=0.8\textwidth]{3122.jpg}
\caption{圆孔频谱}
\end{minipage}
\end{figure}
各方向上都有明暗交替的衍射纹样,符合前面的推理预期.
\subsubsection{正三角孔}
成像,频谱如图.
\begin{figure}[H]
\begin{minipage}[t]{0.5\linewidth}
\centering
\includegraphics[width=0.8\textwidth]{3131.jpg}
\caption{正三角孔像} 
\end{minipage}
\hfill
\begin{minipage}[t]{0.5\linewidth}
\centering
\includegraphics[width=0.8\textwidth]{3132.jpg}
\caption{正三角孔频谱}
\end{minipage}
\end{figure}
对称的六枝明暗交替的衍射纹样,符合前面的推理预期.

由于像与频谱是两次分别拍的,过程中旋转了$180^\circ$,所以没有衍射纹垂直于边.
\subsubsection{五角星孔}
成像,频谱如图.
\begin{figure}[H]
\begin{minipage}[t]{0.5\linewidth}
\centering
\includegraphics[width=0.8\textwidth]{3141.jpg}
\caption{五角星孔像} 
\end{minipage}
\hfill
\begin{minipage}[t]{0.5\linewidth}
\centering
\includegraphics[width=0.8\textwidth]{3142.jpg}
\caption{五角星孔频谱}
\end{minipage}
\end{figure}
对称的十枝明暗交替的垂直于边的衍射纹样,符合前面的推理预期.
\subsubsection{方孔方阵与方孔密排}
方孔方阵频谱如图.
\begin{figure}[H]
\begin{minipage}[t]{0.5\linewidth}
\centering
\includegraphics[width=0.8\textwidth]{3152.jpg}
\caption{方孔方阵频谱}
\end{minipage}
\hfill
\begin{minipage}[t]{0.5\linewidth}
\centering
\includegraphics[width=0.8\textwidth]{3153.jpg}
\caption{方孔方阵频谱缩放}
\end{minipage}
\end{figure}
大尺度的阵列结构特征是二维点阵的点与点之间周期结构的贡献,大尺度的强度分布是小方孔的贡献,小尺度的小方块是整个二维点阵所占方块的贡献.符合前面的推理预期.

方孔密排频谱如图.
\begin{figure}[H]
\begin{minipage}[t]{0.5\linewidth}
\centering
\includegraphics[width=0.8\textwidth]{3154.jpg}
\caption{方孔密排频谱}
\end{minipage}
\hfill
\begin{minipage}[t]{0.5\linewidth}
\centering
\includegraphics[width=0.8\textwidth]{3155.jpg}
\caption{方孔密排频谱缩放}
\end{minipage}
\end{figure}
大尺度的阵列结构特征是二维点阵的点与点之间周期结构的贡献,大尺度的强度分布是小方孔的贡献,小尺度的小方块是整个二维点阵所占方块的贡献.符合前面的推理预期.
\subsubsection{圆孔方阵与圆孔密排}
圆孔方阵频谱如图.
\begin{figure}[H]
\begin{minipage}[t]{0.5\linewidth}
\centering
\includegraphics[width=0.8\textwidth]{3161.jpg}
\caption{圆孔方阵频谱}
\end{minipage}
\hfill
\begin{minipage}[t]{0.5\linewidth}
\centering
\includegraphics[width=0.8\textwidth]{3162.jpg}
\caption{圆孔方阵频谱缩放}
\end{minipage}
\end{figure}
大尺度的阵列结构特征是二维点阵的点与点之间周期结构的贡献,大尺度的强度分布是小圆孔的贡献,小尺度的小方块是整个二维点阵所占方块的贡献.符合前面的推理预期.

圆孔密排频谱如图.
\begin{figure}[H]
\begin{minipage}[t]{0.5\linewidth}
\centering
\includegraphics[width=0.8\textwidth]{3163.jpg}
\caption{圆孔密排频谱}
\end{minipage}
\hfill
\begin{minipage}[t]{0.5\linewidth}
\centering
\includegraphics[width=0.8\textwidth]{3164.jpg}
\caption{圆孔密排频谱缩放}
\end{minipage}
\end{figure}
大尺度的阵列结构特征是二维点阵的点与点之间周期结构的贡献,大尺度的强度分布是小圆孔的贡献,小尺度的小方块是整个二维点阵所占方块的贡献.符合前面的推理预期.

最后补一下方孔方阵、方孔密排、圆孔方阵、圆孔密排的像.
\begin{figure}[H]
\begin{minipage}[t]{0.5\linewidth}
\centering
\includegraphics[width=0.8\textwidth]{3151.jpg}
\caption{方孔方阵像}
\end{minipage}
\hfill
\begin{minipage}[t]{0.5\linewidth}
\centering
\includegraphics[width=0.8\textwidth]{3165.jpg}
\caption{方孔密排像}
\end{minipage}
\end{figure}
\begin{figure}[H]
\begin{minipage}[t]{0.5\linewidth}
\centering
\includegraphics[width=0.8\textwidth]{3166.jpg}
\caption{圆孔方阵像}
\end{minipage}
\hfill
\begin{minipage}[t]{0.5\linewidth}
\centering
\includegraphics[width=0.8\textwidth]{3167.jpg}
\caption{圆孔密排像}
\end{minipage}
\end{figure}
\subsection{空间滤波}
\subsubsection{一维光栅}
频谱与原像如图.
\begin{figure}[H]
\begin{minipage}[t]{0.5\linewidth}
\centering
\includegraphics[width=0.8\textwidth]{3211.jpg}
\caption{一维光栅频谱}
\end{minipage}
\hfill
\begin{minipage}[t]{0.5\linewidth}
\centering
\includegraphics[width=0.8\textwidth]{3212.jpg}
\caption{一维光栅原像}
\end{minipage}
\end{figure}
进行一系列空间滤波后,成像分别如下.
\begin{figure}[H]
\begin{minipage}[t]{0.5\linewidth}
\centering
\includegraphics[width=0.8\textwidth]{3213.jpg}
\caption{一维光栅通$0$级衍射纹}
\end{minipage}
\hfill
\begin{minipage}[t]{0.5\linewidth}
\centering
\includegraphics[width=0.8\textwidth]{3214.jpg}
\caption{一维光栅通$0,\pm 1$级衍射纹}
\end{minipage}
\end{figure}
通0级衍射纹时,成像均匀光斑.通$0,\pm 1$级衍射纹时,成像空间频率与原像相同.符合前面的推理预期.
\begin{figure}[H]
\begin{minipage}[t]{0.5\linewidth}
\centering
\includegraphics[width=0.8\textwidth]{3215.jpg}
\caption{一维光栅挡$\pm 1$级衍射纹}
\end{minipage}
\hfill
\begin{minipage}[t]{0.5\linewidth}
\centering
\includegraphics[width=0.8\textwidth]{3216.jpg}
\caption{一维光栅挡$0$级衍射纹}
\end{minipage}
\end{figure}
挡$\pm 1$级衍射纹时,成像的空间频率为原像的2倍.挡$0$级衍射纹时,由于原像中通光部分更多,出现了衬比度反转,而空间频率不变.符合前面的推理预期.
\subsubsection{二维光栅}
原像与滤波成像结果如图.
\begin{figure}[H]
\begin{minipage}[t]{0.5\linewidth}
\centering
\includegraphics[width=0.8\textwidth]{3221.jpg}
\caption{二维光栅原像}
\end{minipage}
\hfill
\begin{minipage}[t]{0.5\linewidth}
\centering
\includegraphics[width=0.8\textwidth]{3222.jpg}
\caption{二维光栅小孔滤波}
\end{minipage}
\end{figure}
小孔滤波时,成像均匀光斑.符合前面的推理预期.
\begin{figure}[H]
\begin{minipage}[t]{0.5\linewidth}
\centering
\includegraphics[width=0.8\textwidth]{3223.jpg}
\caption{二维光栅水平狭缝滤波}
\end{minipage}
\hfill
\begin{minipage}[t]{0.5\linewidth}
\centering
\includegraphics[width=0.8\textwidth]{3224.jpg}
\caption{二维光栅竖直狭缝滤波}
\end{minipage}
\end{figure}
水平狭缝成像竖条纹,竖直狭缝成像水平条纹.符合前面的推理预期.
\begin{figure}[H]
\centering
\includegraphics[height=4cm]{3225.jpg}
\caption{二维光栅倾斜$45^\circ$狭缝滤波}
\end{figure}
成像几乎为均匀光斑,有垂直于狭缝方向的倾斜特征.符合前面推理的预期.
\subsubsection{“光”字形二维光栅}
原像与滤波成像结果如图.
\begin{figure}[H]
\begin{minipage}[t]{0.5\linewidth}
\centering
\includegraphics[width=0.8\textwidth]{3231.jpg}
\caption{“光”字形二维光栅原像}
\end{minipage}
\hfill
\begin{minipage}[t]{0.5\linewidth}
\centering
\includegraphics[width=0.8\textwidth]{3232.jpg}
\caption{中心小孔$1mm$滤波}
\end{minipage}
\end{figure}
保留中心小孔时,成像有光字形,但是没有网格结构.符合前面推理的预期.
\begin{figure}[H]
\begin{minipage}[t]{0.5\linewidth}
\centering
\includegraphics[width=0.8\textwidth]{3233.jpg}
\caption{中心小孔$0.3mm$滤波}
\end{minipage}
\hfill
\begin{minipage}[t]{0.5\linewidth}
\centering
\includegraphics[width=0.8\textwidth]{3234.jpg}
\caption{相邻小孔$1mm$滤波}
\end{minipage}
\end{figure}
中心小孔缩小时,光字形逐渐模糊,依旧没有网格结构.保留相邻小孔时,成像有光字形,有条纹结构.符合前面推理的预期.
\begin{figure}[H]
\centering
\includegraphics[height=4cm]{3235.jpg}
\caption{相邻小孔$0.3mm$滤波}
\end{figure}
相邻小孔缩小时,保持条纹结构,但光字形逐渐模糊.符合前面推理的预期.

补一下频谱.
\begin{figure}[H]
\begin{minipage}[t]{0.5\linewidth}
\centering
\includegraphics[width=0.8\textwidth]{3236.jpg}
\caption{频谱}
\end{minipage}
\hfill
\begin{minipage}[t]{0.5\linewidth}
\centering
\includegraphics[width=0.8\textwidth]{3237.jpg}
\caption{频谱缩放}
\end{minipage}
\end{figure}
可以看到大尺度的阵列结构特征是二维光栅的贡献,小尺度的特征是光字形的贡献,这里没有小尺度的方块是因为受到了光字形的影响.总之符合前面推理的预期.
\subsubsection{十字光孔}
原像与频谱如图.
\begin{figure}[H]
\begin{minipage}[t]{0.5\linewidth}
\centering
\includegraphics[width=0.8\textwidth]{3241.jpg}
\caption{十字光孔原像}
\end{minipage}
\hfill
\begin{minipage}[t]{0.5\linewidth}
\centering
\includegraphics[width=0.8\textwidth]{3242.jpg}
\caption{十字光孔频谱}
\end{minipage}
\end{figure}
频谱与矩孔衍射十分相似,但小尺度的特征与方孔不同,应当受到了十字形的影响.符合前面推理的预期.
\begin{figure}[H]
\centering
\includegraphics[height=4cm]{3243.jpg}
\caption{十字光孔圆屏滤波}
\end{figure}
成像为十字轮廓.符合前面推理的预期.
\subsection{$\theta$调制}
调制结果如下.
\begin{figure}[H]
\centering
\includegraphics[height=5cm]{331.jpg}
\caption{$\theta$调制}
\end{figure}
实际没有什么特别的现象需要说明.
\subsection{卷积现象}
\begin{figure}[H]
\centering
\includegraphics[height=5cm]{341.jpg}
\caption{二维光栅卷积现象}
\end{figure}
大尺度的特征结构由小光栅常数的二维光栅决定,并随着其转动而转动,小尺度的特征结构由大光栅常数的二维光栅决定,并随着其转动而转动.符合前面推理的预期.

两个光栅常数不同的二维光栅的频谱特征分别如图.
\begin{figure}[H]
\begin{minipage}[t]{0.4\linewidth}
\centering
\includegraphics[width=0.8\textwidth]{342.jpg}
\caption{小光栅常数光栅}
\end{minipage}
\hfill
\begin{minipage}[t]{0.4\linewidth}
\centering
\includegraphics[width=0.8\textwidth]{343.jpg}
\caption{大光栅常数光栅}
\end{minipage}
\end{figure}
一维光栅与二维光栅的卷积如图.
\begin{figure}[H]
\centering
\includegraphics[height=5cm]{344.jpg}
\caption{一维光栅卷积现象}
\end{figure}
也可以由前面的推理得到解释.
\end{document}